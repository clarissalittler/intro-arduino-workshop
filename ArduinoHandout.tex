\documentclass[letterpage,12pt]{article}
\usepackage[urlcolor=magenta]{hyperref}
\usepackage[inline]{enumitem}

\title{Learn Electronics \& Programming with Arduino\vspace{-2ex}}
\date{Updated as of: \today\vspace{-2ex}}
\pagenumbering{gobble}
\renewcommand\abstractname{Summary:}


\begin{document}
\maketitle
\begin{abstract}
  The Arduino microcontroller is a small programmable board that can be used to control all sorts of circuits. In this mini-workshop we'll be learning how to make small circuits, connect them to an Arduino Uno, and program their behavior.
\end{abstract}

\section*{Resources \& useful links}
\begin{description}
  \item[Materials for this class:] \url{https://github.com/clarissalittler/intro-arduino-workshop}
  \item[Built-in tutorials in IDE:] \texttt{File $\rightarrow$ Examples $\rightarrow$ Built-in Examples}
  \item[Built-in tutorials online:] \url{https://www.arduino.cc/en/Tutorial/BuiltInExamples}
  \item[Arduino official tutorial list:] \url{https://www.arduino.cc/en/Tutorial/HomePage}
  \item[TinkerCAD simulator:] \url{https://www.tinkercad.com/#/?type=circuits}
  \item[Arduino IDE download:] \url{https://www.arduino.cc/en/Main/Software}
  \item[Best C programing book:] \url{https://en.wikipedia.org/wiki/The_C_Programming_Language}
  \item[Arduino projects on Instructables] \url{http://www.instructables.com/id/Arduino-Projects/}
  \item[Arduino projects on official site] \url{https://create.arduino.cc/projecthub}
\end{description}

\section*{Getting started at home}
To get started working on your own projects at home you'll need to have, at the bare minimum,
\begin{enumerate*}
  \item An Arduino board
  \item A serial cable to connect it
  \item A computer with the Arduino IDE installed
\end{enumerate*}

You'll probably \textit{also} want some kind of kit with electronics components if you don't have one already. There are a number of them for sale through Adafruit: \url{https://www.adafruit.com/category/17}
\section*{Glossary}
\begin{description}
  \itemsep2pt
  \item[Voltage] The energy differential between two places in a circuit, measured in volts
  \item[Current] How much charge is moving through the circuit per second, measured in Amperes
  \item[Ground] A stable reference point for measuring voltage and a sink for current
  \item[PWM] Pulse-width modulation: a technique for simulating analog output by rapidly turning on and off the digital pin
  \item[Digital pins] Pins that are either \texttt{HIGH} or \texttt{LOW}. Can be used for both input and output.
  \item[Analog pins] Pins that can be read to produce a range of values from 0 to 1023.
  \item[C Programming Language] The underlying language for programming Arduino boards.
  \item[Resistor] A material that \textit{resists} electrical current, slowing it down like a narrow pipe.
  \item[Pot/Potentiometer] Acts like a variable strength resistor
  \item[IDE] Integrated Development Environment, a program that helps you write and run code
  \item[Breadboard] A board that you can plug electronics components into temporarily to build circuits
\end{description}

\end{document}